\chapter{Introdu��o}
\label{chap:Introducao}

Est� � a introdu��o do trabalho. Um bom livro de linguagens de programa��o � o \cite{Sebesta:2005}. 
Conforme Sebesta \citeyearpar{Sebesta:2005}, uma boa linguagem de programa��o � Java \cite{Sun:2010}.

Uma defini��o formal � de sistema de tipos � proposta por Pierce:

\begin{quote}
 \emph{``Um sistema de tipos � um m�todo sint�tico trat�vel para provar a aus�ncia de certos comportamentos em 
   um programa atrav�s da classifica��o de frases de acordo com o tipo de valores que elas computam''} \citep[p. 1]{Pierce:2002}
\end{quote}

Outras refer�ncias: \cite{Alex:2010}, \cite{Weiser:1991} e \cite{norell:thesis}.

\section{Objetivos}
O objetivo deste trabalho � .....

\section{Tabela}
Um exemplo de tabela � a \ref{tab:curry}:

\begin{table}[!hbt] % [htb]-> here, top, bottom
   \centering   % tabela centralizada
   \setlength{\arrayrulewidth}{1\arrayrulewidth}  % espessura da  linha
   \setlength{\belowcaptionskip}{5pt}  % espa�o entre caption e tabela
   \caption{Correspond�ncia \textit{Curry-Howard}}
   \begin{tabular}{l|l} % c=center, l=left, r=right 
      \hline
      \textbf{L�gica} & \textbf{Linguagens de Programa��o} \\
      \hline
      proposi��es & tipos  \\
      proposi��o $P \supset Q$ & tipo $P \rightarrow Q$ (fun��o) \\
      proposi��o $P \wedge Q$ & tipo de produto $P \times Q$\\
      prova de uma proposi��o $P$ & termo $t$ do tipo $P$  (ou seja, $t:P$)\\
      proposi��o $P$ � prov�vel & tipo $P$ � habitado por algum termo \\
      \hline
   \end{tabular}
   \label{tab:curry}
\end{table}

