\chapter{Introdução}
\label{chap:Introducao}

Está é a introdução do trabalho. Um bom livro de linguagens de programação é o \cite{Sebesta:2005}. 
Conforme Sebesta \citeyearpar{Sebesta:2005}, uma boa linguagem de programação é Java \cite{Sun:2010}.

Segundo Lee~\citeyearpar{Lee:2009}, a definição de contexto mais citada na bibliografia é a definição
proposta por Abowd \textit{et al.}:

\begin{quote}
         Contexto é qualquer informação que pode ser utilizada 
         para caracterizar a situação de uma entidade. Uma entidade é uma pessoa, lugar ou objeto que 
         podem ser considerados relevantes para a interação entre um usuário e uma aplicação, 
         incluindo o usuário e as suas próprias aplicações. \citep[tradução nossa]{Abowd:1999}
\end{quote}

Outras referências: \cite{Alex:2010}, \cite{Weiser:1991} e \cite{norell:thesis}.

\section{Objetivos}
O objetivo deste trabalho é .....

\section{Tabela}
Um exemplo de tabela é a \ref{tab:curry}:

\begin{table}[!hbt] % [htb]-> here, top, bottom
   \centering   % tabela centralizada
   \setlength{\arrayrulewidth}{1\arrayrulewidth}  % espessura da  linha
   \setlength{\belowcaptionskip}{5pt}  % espaço entre caption e tabela
   \caption{Correspondência \textit{Curry-Howard}}
   \begin{tabular}{l|l} % c=center, l=left, r=right 
      \hline
      \textbf{Lógica} & \textbf{Linguagens de Programação} \\
      \hline
      proposições & tipos  \\
      proposição $P \supset Q$ & tipo $P \rightarrow Q$ (função) \\
      proposição $P \wedge Q$ & tipo de produto $P \times Q$\\
      prova de uma proposição $P$ & termo $t$ do tipo $P$  (ou seja, $t:P$)\\
      proposição $P$ é provável & tipo $P$ é habitado por algum termo \\
      \hline
   \end{tabular}
   \label{tab:curry}
\end{table}

