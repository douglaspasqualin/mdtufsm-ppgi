%\listfiles
\documentclass[diss]{mdtufsm}
% um tipo específico de monografia pode ser informado como parâmetro opcional:
%\documentclass[tese]{mdtufsm}
% a opção `openright' pode ser usada para forçar inícios de capítulos
% em páginas ímpares
% \documentclass[openright]{mdtufsm}
% para gerar uma versão frente-e-verso, use a opção 'twoside':
% \documentclass[twoside]{mdtufsm}

\usepackage[T1]{fontenc}        % pacote para conj. de caracteres correto
\usepackage{fix-cm} %para funcionar corretamente o tamanho das fontes da capa
\usepackage{times, color, xcolor}       % pacote para usar fonte Adobe Times e cores
\usepackage[utf8]{inputenc}   % pacote para acentuação
\usepackage{graphicx}  % pacote para importar figuras
\usepackage[bottom]{footmisc} %pacote para as notas de rodapé que extrapolam o tamanho da página ficarem posicionadas ao pé da página posterior
\usepackage{amsmath,latexsym,amssymb} %Pacotes matemáticos
\usepackage[%hidelinks%, 
            bookmarksopen=true,linktoc=none,colorlinks=true,
            linkcolor=black,citecolor=black,filecolor=magenta,urlcolor=blue,
            pdftitle={Título da Dissertação ou Trabalho ....},
            pdfauthor={Nome Autor Sobrenome},
            pdfsubject={Dissertação de Mestrado},
            pdfkeywords={Dissertação, Modelo, LaTeX}
            ]{hyperref} %hidelinks disponível no pacote hyperref a partir da versão 2011-02-05  6.82a
%Nesse caso, hidelinks retira os retângulos em volta dos links das referências

%Margens conforme MDT 7ª edição, arrumar diretamente no mdtufsm.cls para funcionar a opção twoside *PENDENTE*
\usepackage[inner=30mm,outer=20mm,top=30mm,bottom=20mm]{geometry} 

%==============================================================================
% Se o pacote hyperref foi carregado a linha abaixo corrige um bug na hora
% de montar o sumário da lista de figuras e tabelas
% Se o pacote não foi carregado, comentar a linha %
%==============================================================================
\input{macros/bugcaption}

%==============================================================================
% Identificação do trabalho
%==============================================================================
\title{Título da Dissertação ou Trabalho de Graduação Conforme a MDT 7ª Edição}

\author{Sobrenome}{Nome Autor}
%Descomentar se for uma "autora"
%\autoratrue

\course{Programa de Pós-Graduação em Informática}
\altcourse{Programa de Pós-Graduação em Informática}

\institute{Centro de Tecnologia}
\degree{Mestre em Ciência da Computação}

% Número do TG (verificar na secretaria do curso)
% Para mestrado deixar sem opção dentro do {}
\trabalhoNumero{}

%Orientador
\advisor[Profª.]{Dr.}{Silva}{João da}
%Se for uma ``orientadora'' descomentar a linha baixo
%\orientadoratrue

%Co orientador, comentar se não existir
\coadvisor[Prof.]{Drª.}{Pereira}{Maria Regina}
\coorientadoratrue %Se for uma ``Co-Orientadora''

%Avaliadores (Banca)
\committee[Dr.]{Sobrenome}{Fulano}{UFSM}
\committee[Dr.]{Sobrenome2}{Fulano2}{INPE}

% a data deve ser a da defesa; se nao especificada, são gerados
% mes e ano correntes
\date{01}{Março}{2011}

%Palavras chave
\keyword{Dissertação} 
\keyword{Modelo}
\keyword{LaTeX}

%%=============================================================================
%% Início do documento
%%=============================================================================
\begin{document}

%%=============================================================================
%% Capa e folha de rosto
%%=============================================================================
\maketitle

%%=============================================================================
%% Catalogação (obrigatório para mestrado) e Folha de aprovação
%%=============================================================================
%Somente obrigatório para dissertação, para TG, remover as linhas	77	%
%Como a CIP vai ser impressa atrás da página de rosto, as margens inner e outer	
%devem ser invertidas.
\newgeometry{inner=20mm,outer=30mm,top=30mm,bottom=20mm}	
\makeCIP{nomedoautor@gmail.com} %email do autor		
\restoregeometry

%Se for usar a catalogação gerada pelo gerador do site da biblioteca comentar as linhas
%acima e utilizar o comando abaixo
%\includeCIP{CIP.pdf}

%folha de aprovação
\makeapprove

%%=============================================================================
%% Dedicatória (opcional)
%%=============================================================================
\clearpage
\begin{flushright}
\mbox{}\vfill
{\sffamily\itshape À UFSM ......}
\end{flushright}

%%=============================================================================
%% Agradecimentos (opcional)
%%=============================================================================
\chapter*{Agradecimentos}
Obrigado ao \LaTeX por facilitar a digitação do trabalho


%%=============================================================================
%% Epígrafe (opcional)
%%=============================================================================
\clearpage
\begin{flushright}
\mbox{}\vfill
{\sffamily\itshape
``Frase da epígrafe'' \\ }
--- \textsc{Autor da frase}
\end{flushright}


%%=============================================================================
%% Resumo
%%=============================================================================
\begin{abstract}
Este é o resumo do trabalho ... Este é o resumo do trabalho ...
Este é o resumo do trabalho ... Este é o resumo do trabalho ...
Este é o resumo do trabalho ... Este é o resumo do trabalho ...
Este é o resumo do trabalho ... Este é o resumo do trabalho ...
Este é o resumo do trabalho ... Este é o resumo do trabalho ...
Este é o resumo do trabalho ... Este é o resumo do trabalho ...
\end{abstract}

%%=============================================================================
%% Abstract
%%=============================================================================
% resumo na outra língua
% como parametros devem ser passados o titulo, o nome do curso,
% as palavras-chave na outra língua, separadas por vírgulas, o mês em inglês
%o a sigla do dia em inglês: st, nd, th ...
\begin{englishabstract}
{Dissertation Title}
{Post-Graduate Program in Informatics}
{Keywords1. Keyword2}
{March}
{st}
Abstract ... Abstract ...Abstract ...Abstract ...Abstract ...Abstract ...Abstract ...
Abstract ...Abstract ...Abstract ...Abstract ...Abstract ...Abstract ...Abstract ...
Abstract ...Abstract ...Abstract ...Abstract ...Abstract ...Abstract ...Abstract ...
Abstract ...Abstract ...Abstract ...Abstract ...Abstract ...Abstract ...Abstract ...
\end{englishabstract}

%% Lista de Ilustrações (opc)
%% Lista de Símbolos (opc)
%% Lista de Anexos e Apêndices (opc)

%%=============================================================================
%% Lista de figuras (comentar se não houver)
%%=============================================================================
%\listoffigures

%%=============================================================================
%% Lista de tabelas (comentar se não houver)
%%=============================================================================
\listoftables

%%=============================================================================
%% Lista de Apêndices (comentar se não houver)
%%=============================================================================
\listofappendix

%%=============================================================================
%% Lista de Anexos (comentar se não houver)
%%=============================================================================
\listofannex

%%=============================================================================
%% Lista de abreviaturas e siglas
%%=============================================================================
 %o parametro deve ser a abreviatura mais longa
\begin{listofabbrv}{UbiComp}
   \item [BNF] \textit{Backus-Naur Form}
   \item [UbiComp] Computação Ubíqua
\end{listofabbrv}


%%=============================================================================
%% Lista de simbolos (opcional)
%%=============================================================================
%Simbolos devem aparecer conforme a ordem em que aparecem no texto
% o parametro deve ser o símbolo mais longo
\begin{listofsymbols}{teste}
  \item [$\varnothing$] vazio
  \item [$\Gamma$]  Gama
  \item [$\forall$] Para todo
\end{listofsymbols}

%%=============================================================================
%% Sumário
%%=============================================================================
\tableofcontents


%%=============================================================================
%% Início da dissertação
%%=============================================================================
\setlength{\baselineskip}{1.5\baselineskip}

%Adiciona cada capitulo
\chapter{Introdu��o}
\label{chap:Introducao}

Est� � a introdu��o do trabalho. Um bom livro de linguagens de programa��o � o \cite{Sebesta:2005}. 
Conforme Sebesta \citeyearpar{Sebesta:2005}, uma boa linguagem de programa��o � Java \cite{Sun:2010}.

Segundo Lee~\citeyearpar{Lee:2009}, a defini��o de contexto mais citada na bibliografia � a defini��o
proposta por Abowd \textit{et al.}:

\begin{quote}
 \emph{``Contexto � qualquer informa��o que pode ser utilizada 
         para caracterizar a situa��o de uma entidade. Uma entidade � uma pessoa, lugar ou objeto que 
         podem ser considerados relevantes para a intera��o entre um usu�rio e uma aplica��o, 
         incluindo o usu�rio e as suas pr�prias aplica��es.''} \citep[tradu��o nossa]{Abowd:1999}
\end{quote}

Outras refer�ncias: \cite{Alex:2010}, \cite{Weiser:1991} e \cite{norell:thesis}.

\section{Objetivos}
O objetivo deste trabalho � .....

\section{Tabela}
Um exemplo de tabela � a \ref{tab:curry}:

\begin{table}[!hbt] % [htb]-> here, top, bottom
   \centering   % tabela centralizada
   \setlength{\arrayrulewidth}{1\arrayrulewidth}  % espessura da  linha
   \setlength{\belowcaptionskip}{5pt}  % espa�o entre caption e tabela
   \caption{Correspond�ncia \textit{Curry-Howard}}
   \begin{tabular}{l|l} % c=center, l=left, r=right 
      \hline
      \textbf{L�gica} & \textbf{Linguagens de Programa��o} \\
      \hline
      proposi��es & tipos  \\
      proposi��o $P \supset Q$ & tipo $P \rightarrow Q$ (fun��o) \\
      proposi��o $P \wedge Q$ & tipo de produto $P \times Q$\\
      prova de uma proposi��o $P$ & termo $t$ do tipo $P$  (ou seja, $t:P$)\\
      proposi��o $P$ � prov�vel & tipo $P$ � habitado por algum termo \\
      \hline
   \end{tabular}
   \label{tab:curry}
\end{table}


\chapter{Desenvolvimento}

Este � o desenvolvilmento ...

\section{Se��o1}
ssss

\subsection{Subse��o1}
xxxx

\subsection{Subse��o2}
vvvvv

\subsubsection{Subsubse��o1}
Este tipo de se��o n�o vai para o sum�rio.
\include{capitulos/conclusao}

\setlength{\baselineskip}{\baselineskip}

%%=============================================================================
%% Referências
%%=============================================================================
\bibliographystyle{abnt}
\bibliography{referencias/referencias}



%IMPORTANTE: Se precisar usar alguma seção ou subseção dentro dos apêndices ou
%anexos, utilizar o comando \tocless para não adicionar no Sumário
%Exemplos: 
% \tocless\section{Histórico}
%%=============================================================================
%% Apêndices
%%=============================================================================
\appendix

\chapter{T�tulo do ap�ndice}
Est� � o ap�ndice A
\include{capitulos/apendiceb}

%%=============================================================================
%% Anexos
%%=============================================================================
\annex

\chapter{T�tulo do Anexo}
Est� � o anexo A

\end{document}
